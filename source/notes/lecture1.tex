% First meeting
% 08-04-2023

\section{Introduction to Linear Systems}
    \renewcommand{\leftmark}{August 09, 2023}
    \pagenumbering{arabic}

    \begin{example}
        Consider the following system of equations:

        \[\begin{rcases}
        x_1 - x_2 &= 3 \\
        x_1 + x_2 &= 4
        \end{rcases}\]

        We can see that \(x_1 = 3.5\) and \(x_2 = 0.5\) solves the system. 

    \end{example}

    \begin{note}
        System of equations with a solution are \emph{consistent}.
    \end{note}

    \begin{example}
        Consider another system of equations:

        \[\begin{rcases}
            2x_1 - 5x_2 &= 8 \\
            4x_1 - 10x_2 &= 9
        \end{rcases}\]

        Observe that both sides of the first equation can be multiplied by 2 to get \(4x_1 - 10x_2 = 16\) which, together with the second equation, yields \(16 = 9\). This is clearly a contradiction, hence, there are no solutions to this system.
    \end{example}

    \begin{note}
        System of equations with no solution are called \emph{inconsistent}.
    \end{note}

    \begin{dfn}[System of equations]
        A system of \(m\) linear equations in \(n\) unknowns \(x_1, x_2, \ldots, x_n\) is a set of \(m\) equations each with \(n\) unknowns. It is usually written as 

        \[\renewcommand{\arraystretch}{1.15}\left.\begin{array}{ccccccccc}
            a_{11}x_1 &+& a_{12}x_2 &+& \cdots &+& a_{1n}x_n &=& b_1 \\
            a_{21}x_1 &+& a_{22}x_2 &+& \cdots &+& a_{2n}x_n &=& b_2 \\
            \vdots &\phantom{+}& \vdots &\phantom{+}& \ddots &\phantom{+}& \vdots &\phantom{=}& \vdots \\
            a_{m1}x_1 &+& a_{m2}x_2 &+& \cdots &+& a_{mn}x_n &=& b_m \\
        \end{array}\right\}\]

        where \(a_{ij}\) is the coefficient of \(x_j\) in the \(i\)th equation.
    \end{dfn}

    \begin{dfn}[Solution]
        A solution to \((*)\) is a sequence of \(n\) numbers \((s_1, s_2, \ldots, s_n)\) such that each equation in \((*)\) is satisfied if \(x_i = s_i\) for each \(i = 1, 2, \ldots, n\).
    \end{dfn}

    \begin{dfn}[Trivial solution]
        The solution set \((0, 0, \ldots, 0)\) is called the trivial solution.
    \end{dfn}

    \begin{dfn}[Homogeneous system]
        If a system \((*)\) has \(b_i = 0\) for each \(i = 1, 2, \ldots, n\), then \((*)\) is called a homogeneous system.
    \end{dfn}

    \begin{note}
        Homogeneous systems are always consistent since it is satisfied by the trivial solution.
    \end{note}

    \begin{dfn}[Equivalent systems]
        Two systems are equivalent if they have the same number of equations and unknowns, and they have exactly the same set of solutions.
    \end{dfn}

    \begin{dfn}[Elimination method]
        This method is used to solve systems of linear equations. The possible steps are as follows:
        \begin{itemize}
            \item Interchange the \(i\)th and \(j\)th equation.
            \item Multiply an equation by a nonzero constant.
            \item Replace the \(i\)th equation by \(c\) times the \(j\)th equation plus the \(i\)th equation.
        \end{itemize}
    \end{dfn}

    \begin{dfn}[Matrix]
        An \(m \times n\) matrix is a rectangular array of \(mn\) real (or complex) numbers arranged in \(m\) horizontal rows and \(n\) vertical columns:
        \[A = \begin{bmatrix}
            a_{11} & a_{12} & \cdots & a_{1n} \\
            a_{21} & a_{22} & \cdots & a_{2n} \\
            \,\vdots & \,\vdots & \ddots & \,\vdots \\
            a_{m1} & a_{m2} & \cdots & a_{mn}
        \end{bmatrix}\] where the \(i\)th row of \(A\) is \[\begin{bmatrix}
            a_{i1} & a_{i2} & \cdots & a_{in}
        \end{bmatrix}\] and the \(j\)th row of \(A\) is \[\begin{bmatrix}
            a_{1j} \\ a_{2j} \\ \,\vdots \\ a_{mj}
        \end{bmatrix}.\]
    \end{dfn}

    \begin{dfn}[Row and column of a matrix]
        We define \(\row_i(A)\) as the \(i\)th row of \(A\), that is, \[\row_i(A) = \begin{bmatrix}
            a_{i1} & a_{i2} & \cdots & a_{in}
        \end{bmatrix}\] and \(\col_j(A)\) as the \(j\)th column of \(A\), that is, \[\col_j(A) = \begin{bmatrix}
            a_{1j} \\ a_{2j} \\ \,\vdots \\ a_{mj}
        \end{bmatrix}.\]
    \end{dfn}

    \begin{dfn}[Matrix size]
        An \(m \times n\) matrix \(A\) is said to have a size of \(m \times n\). If \(m = n\), then \(A\) is a square matrix of order \(n\).
    \end{dfn}

    \begin{dfn}[Main diagonal]
        In a square matrix of order \(n\) \(A\), the numbers \(a_{ii}\) for \(i = 1, 2, \ldots, n\) form the main diagonal of \(A\). It is usually written as \((a_{11}, a_{22}, \ldots, a_{nn})\).
    \end{dfn}

    \begin{dfn}[\(n\)-vector]
        An \(n\)-vector \(K\) is an \(n \times 1\) matrix.
    \end{dfn}

    \begin{dfn}[Equality of matrices]
        Two matrices \(A = [a_{ij}]\) and \(B = [b_{ij}]\) are said to be equal if they have the same size, that is, they are both \(m \times n\), and
        \(a_{ij} = b_{ij}\) for each \(1 \leq i \leq m\) and \(1 \leq j \leq n\).
    \end{dfn}

    \begin{dfn}[Matrix addition]
        If \(A = [a_{ij}]\) and \(B = [b_{ij}]\) are both \(m \times n\) matrices, then the sum \(A + B\) is the \(m \times n\) matrix \(C = [c_{ij}]\) such that \(c_{ij} = a_{ij} + b_{ij}\) for all \(i, j\).
    \end{dfn}

    \begin{example}
        Let \(\displaystyle A = \begin{bmatrix}
            1 & 0 & 3 \\ 2 & 1 & 4
        \end{bmatrix}\) and \(B = \begin{bmatrix}
            5 & -1 & -3 \\ 4 & 2 & -4
        \end{bmatrix}\). Then, \(A + B = \begin{bmatrix}
            6 & -1 & 0 \\ 6 & 3 & 0
        \end{bmatrix}.\)
    \end{example}

    \begin{dfn}[Scalar multiplication]
        If \(A = [a_{ij}]\) is an \(m \times n\) matrix and \(r\) is a real number, then the scalar multiple of \(A\) by \(r\), denoted by \(rA\), is the \(m \times n\) matrix \(C = [c_{ij}]\) such that \(c_{ij} = ra_{ij}\) for all \(i, j\).
    \end{dfn}

    \begin{example}
        \begin{align*}
            3\begin{bmatrix}
                1 & 2 & 6 \\ 1 & 1 & 0
            \end{bmatrix} &= \begin{bmatrix}
                3 & 6 & 18 \\ 3 & 3 & 0
            \end{bmatrix} \\ 
            \begin{bmatrix}
                8 & 0 & 0 \\ 0 & 8 & 0 \\ 0 & 0 & 8
            \end{bmatrix} &= 8 \begin{bmatrix}
                1 & 0 & 0 \\ 0 & 1 & 0 \\ 0 & 0 & 1
            \end{bmatrix}
        \end{align*}
    \end{example}

    \begin{dfn}[Matrix subtraction]
        Let \(A\) and \(B\) be \(m \times n\) matrices. The difference of \(A\) and \(B\), denoted by \(A - B\), is defined by \(A + (-1)B\).
    \end{dfn}