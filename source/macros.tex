\usepackage{
    enumitem,
    amsmath, amsthm, amssymb,
    geometry,
    fancyhdr,
    marginnote,
    titlesec,
    indentfirst,
    parskip,
    hyperref,
    thmbox, 
    graphicx,
	  mathtools,
    setspace
    % pgfplots, tikz
}

\allowdisplaybreaks
\onehalfspacing

% geometry package
\geometry{
    paper=a4paper,
    twoside,
    outer=1in,
    top=1.25in,
    bottom=1.25in,
    inner=1.3in,
    % showframe
} 

% fancyhdr package
\pagestyle{fancy}

\renewcommand{\sectionmark}[1]{\markboth{}{Lecture \thesection: #1}}

\setlength{\headheight}{13.6pt}
\fancyhead[RO, LE]{\textrm{\rightmark}}
\fancyhead[LO, RE]{\leftmark}
\renewcommand{\headrulewidth}{0.75pt}
\renewcommand{\footrulewidth}{0.75pt}

% titlesec package
\titleformat{\section}{\Large\bfseries}{Lecture~\thesection:}{1ex}{}

% parskip package
\setlength{\parindent}{0.25in}
\setlength{\parskip}{10pt}

% hyperref package
\hypersetup{
    colorlinks,
    pdftitle=Linear Algebra,
    pdfauthor=Kevin,
}


% misc

\renewcommand{\emph}{\textsl}

\setcounter{tocdepth}{1}
\newcommand\xqed[1]{%
\leavevmode\unskip\penalty9999 \hbox{}\nobreak\hfill
\quad\hbox{#1}}
\newcommand\quod{\xqed{$\square$}}

\newenvironment{solution}[1][\undefined]
  {\ifx\undefined#1 Solution:\ \else Solution #1:\ \fi}
  {\quod}
\newenvironment{recall}
  {\noindent\small\textbf{Recall :}\ \normalfont}
  {}

\counterwithin{figure}{section}
% \counterwithin{footnote}{chapter}

\makeatletter
\let\oldref=\ref
\def\ref#1{\ifcsname r@#1\endcsname \oldref{#1}\else #1\fi}
\makeatother

\newcommand\fnote[1]{%
  \begingroup
  \renewcommand\thefootnote{}\footnote{#1}%
  \addtocounter{footnote}{-1}%
  \endgroup
}

\newcommand{\lowerparen}[2]{%
  \raisebox{-#1}{\(\displaystyle\left(\raisebox{#1}{\(\displaystyle #2\)}\right)\)}}

\newcommand{\sectionbreak}{\clearpage}

% Math commands and environments
% \renewenvironment{bmatrix}[1][r]{\begin{bmatrix*}[#1]}{\end{bmatrix*}}
\newtheorem[M]{thm}{Theorem}[section]
\newtheorem[M]{dfn}{Definition}[section]
\newtheorem[S]{note}{Note}[section]

\thmboxoptions{cut=true, bodystyle=\normalfont, leftmargin=7mm, rightmargin=5mm, headstyle={#1 #2 \normalfont}, titlestyle={ (#1)}, vskip=1.5mm}

\newcommand{\rtypeI}[2]{#1 \leftrightarrow #2}
\newcommand{\rtypeII}[2]{#1#2 \to #2}
\newcommand{\rtypeIII}[2]{#1 + #2 \to #2}

\newcommand{\typeI}[3]{#1_{\rtypeI{#2}{#3}}}
\newcommand{\typeII}[3]{#1_{\rtypeII{#2}{#3}}}
\newcommand{\typeIII}[3]{#1_{\rtypeIII{#2}{#3}}}

\newcommand{\requiv}{\stackrel{\text{row}}{\sim}}
\newcommand{\cequiv}{\stackrel{\text{col}}{\sim}}
% \renewcommand{\implies}{\Rightarrow}

\newcommand{\real}{\mathbb{R}}

\DeclareMathOperator{\row}{row}
\DeclareMathOperator{\col}{col}
\DeclareMathOperator{\Tr}{Tr}